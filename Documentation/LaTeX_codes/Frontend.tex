\documentclass[12pt,a4paper]{article}
\usepackage[utf8]{inputenc}
\usepackage{amsmath}
\usepackage{amsfonts}
\usepackage{amssymb}
\usepackage{float}
\usepackage{enumerate}
\usepackage{graphicx}
%\usepackage{gensymb}
\usepackage[none]{hyphenat}
\usepackage[left=1.5cm,right=1.5cm,top=1cm,bottom=1cm]{geometry}
\pagenumbering{gobble}

\title{\textbf{Project - Inventory Tracking System(ITS)}}
\author{P.Anudeep Rao\\ P.Rishi Manoj\\ J.Varshini}
\date{May 22, 2022}

\begin{document}
    \maketitle
    \begin{Large}
     \centering 
     \section*{Front End Implementation Details}
    \end{Large}
    
    \begin{enumerate}
    \item When opened, different roles are displayed.
    \begin{itemize}
    \item Admin
    \item Operations Manager
    \item Supervisor
    \item Faculty
    \end{itemize}
    \item A role is selected, and credentials are to be entered.
    \end{enumerate}
    
    \newpage 
    
    \section{Admin Mode:}
    \begin{enumerate}
    \item Appoint Operations Manager for the fields.
    \begin{itemize}
    \item The fields are displayed.
    \item Select the field for which you want to appoint the Operation Manager.
    \item The details of the Operations Manager is stored in the database by giving required info.
    \end{itemize}
    
    \item View the profile of Operation Manager
    \begin{itemize}
    \item The list of Operations Managers is shown.
    \item Select the Operation Manager to know the info of OM.
    \end{itemize}
    
    \item Remove Operation Manager
    \begin{itemize}
    \item List of Operation Managers is shown.
    \item Select the Operation Manager whom you want to remove.(The total data of the person is removed)
    \end{itemize}
    
    \item Providing the details of sellers
    \begin{itemize}
    \item The Seller Info is to be given by the Admin and is stored in the database.
    \end{itemize}
        
    \item Accessing all the info of the Department.
    \begin{itemize}
    \item All the fields are listed.
    \item A field is selected then the view is changed which is similar to that of the operations manager.
    \end{itemize}
    
    \item Change Password
    \begin{itemize}
    \item Enter Old Password.
    \item Enter New Password.
    \item Confirm New Password.(The new password is updated in the database.)
    \end{itemize}
    \end{enumerate}
    
    \newpage
    
\section{Operations Manager Mode:}

	\begin{enumerate}
	\item Add new Object classes.
	\begin{itemize}
	\item Enter the name of the object class which you want to create.(The object class is created.)
	\end{itemize}
	
	\item Delete object classes.
	\begin{itemize}
	\item List of the object classes is shown.
	\item Select the object class and click delete button.(The object class data is deleted.)
	\end{itemize}
	
	
	\item Appoint Supervisor for Object Class(es)
	\begin{itemize}
		\item The info of the Supervisor is given by the OM and the object class for which he is Supervisor, this info is stored in the database.
	\end{itemize}
	
	\item View Supervisor Info
	\begin{itemize}
	\item List of all Supervisors in Operation Manager's Field is shown.
	\item Select a supervisor
	\item Info of the supervisor is displayed. 
	\end{itemize}
	
	\item View Object instance Info
	\begin{itemize}
	\item All the object instances in the object classes are shown which can be filtered by the OM for viewing.
	\item Select one of the object classes. 
	\item After selecting one object instance, the whole info of the object instance is displayed.
	\end{itemize}
	
	\item Change Password
    \begin{itemize}
    \item Enter Old Password.
    \item Enter New Password.
    \item Confirm New Password.(The new password is updated in the database.)
    \end{itemize}
	\end{enumerate}
    
    \newpage
    
    \section{Supervisor Mode:}
    \begin{enumerate}
    \item Add object instances and store info of a bought item.
    \begin{itemize}
    \item The Item Info is asked which is given by the Supervisor. (Every object is assigned an IITH ID and the count of the object class is kept track.) 
    \end{itemize}
    
    \item Delete object instances
    \begin{itemize}
	\item List of the object instances is shown.
	\item Select the object instance.
	\item Then click on the delete button, the instance is deleted.
	\end{itemize}
	
	\item Update status of an object instance. (Working/Damaged/Under Repair)			
	\begin{itemize}
	\item List of the object instances is shown.
	\item Select the object instance.
	\item Then, the status of the instance can be updated.
	\end{itemize}
	
	\item Access Info of Object classes assigned to the Supervisor.
	\begin{itemize}
	\item All the object instances in the object classes under the  Supervisor are shown.
	\item Select any of the object instance.
	\item After Selecting, the whole info of the object instance is displayed.
	\end{itemize}
	
	\item Change Password
    \begin{itemize}
    \item Enter Old Password.
    \item Enter New Password.
    \item Confirm New Password.(The new password is updated in the database.)
    \end{itemize}
    \end{enumerate}
    
    \newpage
    \section{Faculty}
    \begin{enumerate}
    \item Select the field about which Faculty wants the info.
    \begin{itemize}
    \item Lab
    \item Library
    \item Classroom
    \end{itemize}
    
    \item After selecting, 
    \begin{itemize}
	\item All the object instances in the object classes are shown which can be filtered by the faculty for viewing.
	\item Select one of the object instance. 
	\item After selecting one object instance, the whole info of the object instance is displayed.
    \end{itemize}
    
    \item Change Password
    \begin{itemize}
    \item Enter Old Password.
    \item Enter New Password.
    \item Confirm New Password.(The new password is updated in the database.)
    \end{itemize}
    \end{enumerate}
\end{document}